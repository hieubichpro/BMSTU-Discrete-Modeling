\chapter{Задание}

Реализовать программу с графическим интерфейсом для генерации последовательностей псевдослучайных чисел с использованием табличного и алгоритмического способов и определения коэффициента случайности полученных и введенных последовательностей.

\section{Табличный способ}

Источником случайных чисел при работе табличного генератора является таблица случайных чисел, расположенная в памяти ЭВМ. 

При реализации табличного способа в лабораторной работе использовалась таблица из книги случайных чисел <<A Million Random Digits with 100,000 Normal Deviates>> организации RAND. Обход таблицы выполнялся слева направо сверху вниз.

\section{Алгоритмический способ}

Работа алгоритмического генератора основана на создании случайных чисел с помощью определенных алгоритмов.

При реализации алгоритмического способа в лабораторной работе использовался линейный конгруэнтный метод. В указанном методе каждое новое число определяется предшествующим числом в соответствии с формулой:

\begin{equation}
	y_{n + 1} = (a \cdot y_{n} + \mu)\mod m, n >= 0,
\end{equation}

\noindentгде $a > 0$ --- множитель, $\mu >= 0$ --- приращение, $m > 0$ --- модуль. В данной лабораторной работе использовались следующие значения: $a = 84589, \mu = 45989, m = 217728$.

\section{Критерий оценки случайности}

При реализации лабораторной работы был составлен критерий оценки случайности, при котором коэффициент случайности определяется следующим образом:

\begin{enumerate}
	\item Вычисляется список $differences$ модулей разности чисел, расположенных рядом:
		\begin{equation}
			difference_{i} = |number_{i + 1} - number_{i}|.
		\end{equation}
	\item Элементы списка $differences$ увеличиваются на единицу.
%	\item Определяется максимальное значение $max\_difference$ элементов списка $differences$.
	\item Вычисляется список $ratios$ отношений элементов списка $differences$ к значению $max\_difference$:
		\begin{equation}
			ratio_{i} = \frac{difference_{i}}{max\_difference}.
		\end{equation}
	\item Коэффициент случайности $chance$ определяется как среднее арифметическое элементов списка $ratios$.
\end{enumerate}

Коэффициент случайности может принимать значения из интервала $(0;1]$. При коэффициенте случайности, равном единице, последовательность не является случайной. Последовательность стремится к случайной при ближении коэффициента случайности к нулю.

\chapter{Реализация}

\section{Детали реализации}

На листинге \ref{lst:tabular.py} показана реализация функции генерации последовательности псевдослучайных чисел с использованием табличного способа.

\includelistingpretty
    {tabular.py}
    {Python}
    {Генерация последовательности псевдослучайных чисел с использованием табличного способа}
\newpage
На листинге \ref{lst:algorithmic.py} представлена реализация функции генерации последовательности псевдослучайных чисел с использованием линейного конгруэнтного метода.

\includelistingpretty
    {algorithmic.py}
    {Python}
    {Генерация последовательности псевдослучайных чисел с использованием линейного конгруэнтного метода}
    
На листинге \ref{lst:chance.py} показана реализация функции определения коэффициента случайности последовательности псевдослучайных чисел.

\includelistingpretty
    {chance.py}
    {Python}
    {Определение коэффициента случайности последовательности псевдослучайных чисел}


\section{Полученный результат}

На рисунке \ref{img:generation} представлена страница программы со сгенерированными последовательностями псевдослучайных чисел с использованием табличного и алгоритмического способов и определенными коэффициентами случайности полученных последовательностей.

\includeimage
    {generation}
    {f}
    {h}
    {1.0\textwidth}
    {Табличный и алгоритмический генераторы}
    
На рисунке \ref{img:manual} показаны части страницы программы с определенными коэффициентами случайности следующих введенных последовательностей:
    
\begin{enumerate}
	\item Введена строго возрастающая последовательность чисел. Коэффициент равен единице: последовательность не является случайной.
	\item Введена последовательность, состоящая из одинаковых чисел. Коэффициент равен единице: последовательность не является случайной.
	\item Введена строго убывающая последовательность чисел. Коэффициент равен единице: последовательность не является случайной.
	\item Введена последовательность, состоящая из одинаковых пар чисел. Коэффициент равен единице: последовательность не является случайной.
	\item Введена последователность чисел от 0 до 9 в случайном порядке. Коэффициент равен 0.65278: последовательность стремится к случайной.
\end{enumerate}
    
\includeimage
    {manual}
    {f}
    {h}
    {0.8\textwidth}
    {Определение коэффициентов случайности последовательности}
